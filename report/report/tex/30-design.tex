\chapter{Конструкторский раздел}
\label{cha:design}

В данном разделе описано проектирование метода рендеринга воды с помощью вокселов
с указанием соответствующих схем алгоритмов.
% схемы алгоритмов, расписать IDEF0, спускаясь к деталям каждого этапа

\section{Архитектура приложения}

Исходя из поставленной цели, разработанное приложение должно решать задачу
синтеза сложного динамического изображения.

Эту задачу принято разделять на следующие этапы:
\begin{itemize}
    \item разработка трёхмерной математической модели синтезируемой визуальной
    обстановки;
    \item задание положения наблюдателя, картинной плоскости, размеров окна
    вывода, значений управляющих сигналов;
    \item определение операторов, осуществляющих пространственное перемещение
    объектов визуализации;
    \item преобразования координат объектов в координаты наблюдателя;
    \item отсечение объектов сцены по границам пирамиды видимости;
    \item вычисление двумерных перспективных проекций объектов на картинную плоскость;
    \item удаление невидимых линий и поверхностей при заданном положении наблюдателя;
    \item закрашивание и затенение видимых объектов сцены;
    \item вывод полученного полутонового изображения на экран растрового дисплея.
\end{itemize}

Далее приводятся решения для каждого этапа синтеза изображения.

\subsection{Модуль }
% Вывод по разделу
\section{Вывод}


%%% Local Variables:
%%% mode: latex
%%% TeX-master: "rpz"
%%% End:
