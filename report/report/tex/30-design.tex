\chapter{Конструкторский раздел}
\label{cha:design}

В данном разделе описано проектирование метода рендеринга воды с помощью вокселов
с указанием соответствующих схем алгоритмов.

Разработанное приложение должно решать задачу синтеза сложного динамического изображения.

Эту задачу принято разделять на следующие этапы:
\begin{itemize}
    \item разработка трёхмерной математической модели синтезируемой визуальной
    обстановки;
    \item задание положения наблюдателя, картинной плоскости, размеров окна
    вывода, значений управляющих сигналов;
    \item определение операторов, осуществляющих пространственное перемещение
    объектов визуализации;
    \item преобразования координат объектов в координаты наблюдателя;
    \item отсечение объектов сцены по границам пирамиды видимости;
    \item вычисление двумерных перспективных проекций объектов на картинную плоскость;
    \item удаление невидимых линий и поверхностей при заданном положении наблюдателя;
    \item закрашивание и затенение видимых объектов сцены;
    \item вывод полученного полутонового изображения на экран растрового дисплея.
\end{itemize}
% схемы алгоритмов, расписать IDEF0, спускаясь к деталям каждого этапа

\section{Архитектура приложения}

В соответствии с каждым этапом синтеза изображения были выделены следующие модули программы:
\begin{itemize}
    \item наблюдатель (камера) и его управляющие сигналы;
    \item модуль построения виртуального мира, включающий математической модели воды и земли;
    \item модуль пользовательского интерфейса;
    \item модуль рендеринга сцены.
\end{itemize}

\subsection{Модуль наблюдателя}

Структура данных наблюдателя содержит в себе вектор позиции в трёхмерном пространстве и углы
крена, тангажа и рысканья.

Для наблюдения сцены строится одноточечная перспективная проекция с помощью следующей матрицы:
\begin{math}
    \begin{pmatrix}
    \frac{near \times 2}{right - left} & 0 & \frac{right + left}{right - left} & 0\\
      0 & \frac{near \times 2}{top - bottom} & \frac{top + bottom}{top - bottom} & 0\\
      0 & 0 & -\frac{far + near}{far - near} & -\frac{far \times 2 \times near}{far - near}\\
     0 & 0 & -1 & 0
    \end{pmatrix}
\end{math},где $fovy = 0.785$,
$aspect-ratio = \frac{4}{3}$,
$near = 1$,
$far = 80$,
$top = near \times tg(\frac{fovy}{2})$,
$left = -right$,
$right = top \times aspect-ratio$,
$bottom = -top$.

\subsection{Модуль построения виртуального мира}

Для построения виртуальной поверхности земли предлагается использовать трёхмерный шум Перлина.
Поскольку земля может не менять свою форму во время симуляции, то следует
генерировать данную поверхность однократно.

Для воды необходимо определить клеточный автомат. Он должен учитывать скорость частиц воды,
давление и сопротивление среды($friction$). При этом должны учитываться коллизии с землёй($terrain$). Для этого выделим два двумерных массива размерности мира: в одном содержится текущий уровень воды в данной точке $(water)$, а в другом - кинетическая энергия воды в данной точке ($energy$). Тогда для каждой ячейки $(i, j)$ автомата во время перехода в очередное состояние выполняются следующие действия:
\begin{enumerate}
    \item определить давление с четырёх сторон ($left$, $right$, $front$, $back$)ячейки (если ячейки нет, то давление равно нулю):
    \begin{enumerate}
        \item $left = terrain_{i-1, j} + water_{i-1, j} + energy_{i-1, j}$;
        \item $right = terrain_{i+1, j} + water_{i+1, j} - energy_{i+1, j}$;
        \item $front = terrain_{i, j+1} + water_{i, j+1} - energy_{i, j+1}$;
        \item $back = terrain_{i, j-1} + water_{i, j-1} + energy_{i, j-1}$;
    \end{enumerate}
    \item определить величину изменения уровня воды в данной ячейке на основе заданных давлений,
    сопротивления среды и уровня земли;
    \item обновить состояние клеточного автомата.
\end{enumerate}

\subsection{Модуль пользовательского интерфейса}

В пользовательском интерфейсе необходимо реализовать:
\begin{itemize}
    \item холст для отображения картинной плоскости;
    \item элементы интерфейса для управления наблюдателем.
\end{itemize}

\subsection{Модуль рендеринга сцены}

В данном подразделе описаны основные алгоритмы, используемые для объёмного рендеринга
воксельной сцены.

\subsubsection{Оптимизированный алгоритм марширующих кубов}

На вход алгоритму подают последовательность вокселов, на выходе -
последовательность треугольников на поверхности заданной структуры.

Алгоритм выглядит следующим образом:
\begin{enumerate}
    \item заполнить хеш-таблицу признаком присутствия вокселов;
    \item для каждого присутствующего воксела в сцене определить внешние поверхности;
    \item если внешние поверхности соседних вокселов лежат в одной плоскости - объединить их;
    \item полученный набор поверхностей триангулировать, на основе данных о границах и нормалях к поверхностям.
\end{enumerate}

\subsubsection{Алгоритм с использованием Z-буферизации}

Задаётся массив размерности картинной плоскости для хранения глубины пикселя.
Изначально он инициализирован минимальным значением типа данных.

В процессе рендеринга треугольников для каждого треугольника определяются
прямоугольные объемлющие оболочки.

По всем пикселям внутри оболочки выполяются следующие действия:
\begin{enumerate}
    \item определить глубину пикселя в данной точке из барицентрических координат;
    \item если глубина пикселя меньше текущего значения в массиве глубин, то
    в буфер изображения записываем вычисленный цвет треугольника, а в буфер глубины - новое значение глубины.
\end{enumerate}

\subsubsection{Закрашивание по Ламберту}

На вход алгоритм получает позицию вершины, цвет её материала и нормаль.
На выходе получаем цвет RGBA.

Расчёт интенсивности в вершине производится по формуле: $I = I_{src} * cos(\alpha)$, где $I_{src}$ - интенсивность источника, $\alpha$ - угол между нормалью и вектором направления света.

Интенсивность источника состоит из двух компонент - интенсивности зеркального отражения и интенсивности диффузного.

К интенсивности добавляется интенсивность внешнего постоянного света.

Посредством умножения составляющих частей (красной, зелёной и синей) света на соответствующие части
цвета материала получаем результирующий цвет.
% Вывод по разделу
\section{Вывод}

В данной части были определены основные модули программы, которые необходимо реализовать, и их состав.
%%% Local Variables:
%%% mode: latex
%%% TeX-master: "rpz"
%%% End:
