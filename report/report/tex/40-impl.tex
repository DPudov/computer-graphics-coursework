\chapter{Технологический раздел}
\label{cha:impl}

В данном разделе описаны требуемые средства и подходы к реализации ПО по ранее указанным методам.

% самые важные листинги
% технологические куски
% изящные вещи
% выбор средств реализации: ЯП, ОС, библиотеки
% readme, инструкции запуска-удаления и т.п.

\section{Требования к программному обеспечению}

Разработанное ПО должно моделировать движение воды с использованием вокселов.

Моделирование движения должно осуществляться с использованием операций переноса, масштабирования и поворота.

\section{Используемые технологии}

Для реализации ПО выбран язык Clojure. Данный язык программирования является
компилируемым и, одновременно с этим, динамическим. Clojure - преимущественно
язык функционального программирования, который поддерживает нативный доступ к
Java фреймворкам \cite{site:clojure}.

Данные свойства языка позволяют вести разработку приложений с помощью REPL
(Read-Eval-Print Loop), модифицируя их во время выполнения. Это может быть
полезно для добавления новых функциональностей в программу без её повторной сборки
\cite{site:repl}.
%cite

Для создания пользовательского интерфейса используется Java библиотека Swing.
Swing предоставляет широкий набор компонентов для создания графического пользовательского
интерфейса. Данные компоненты полностью реализованы на Java, поэтому их внешний вид не зависит
от платформы.

Для организации проекта на Clojure и для автоматизации сборки используется Leiningen.
Все конфигурации для сборки приложения, запуска REPL и описания зависимостей управляются с помощью
этого модуля\cite{site:lein}. Результатом сборки является JAR файл, который можно запустить
на любом компьютере, где установлено соответствующее JRE.

Для оптимизации работы алгоритмов рендеринга использованы подсказки типа - механизм
языка Clojure, позволяющий избежать рефлексивного обращения к JVM. Также используются
средства параллельной обработки ленивых последовательностей Clojure - "pmap".

Для оптимизации хранения некоторых воксельных структур использована мемоизация -
механизм Clojure, который позволяет запоминать результаты выполнения функций.

\section{Листинги кода}

В данном разделе будут представлены листинги реализации наиболее существенных
модулей программы.

\subsection{Клеточный автомат}

В данном подразделе указан листинг кода, относящийся к реализации клеточного автомата
симуляции воды.

В листинге \ref{lst:cellular} представлена реализация обновления состояния клеточного автомата.

\begin{lstlisting}[
                basicstyle={\fontsize{9}{10}\ttfamily},
                mathescape=true,
                showstringspaces=false,
                tabsize=2,
                flexiblecolumns=true,
                language={Clojure},
                label={lst:cellular}
            ]
(let [left-pressure (if (> i 0)
                                (+ (aget terrain-row-prev j)
                                   (aget water-row-prev j)
                                   (aget energy-row-prev j))
                                0.0)
                right-pressure (if (< i (dec dim))
                                 (+ (aget terrain-row-next j)
                                    (aget water-row-next j)
                                    (- (aget energy-row-next j)))
                                 0.0)
                back-pressure (if (> j 0)
                                (+ (aget terrain-row-cur (dec j))
                                   (aget water-row-cur (dec j))
                                   (aget energy-row-cur (dec j)))
                                0.0)
                front-pressure (if (< j (dec dim))
                                 (+ (aget terrain-row-cur (inc j))
                                    (aget water-row-cur (inc j))
                                    (- (aget energy-row-cur (inc j))))
                                 0.0)]
            (if (and (> i 0)
                     (> (+ (aget terrain-row-cur j)
                           (aget water-row-cur j)
                           (aget energy-row-cur j))
                        left-pressure))
              (let [flow (/ (min (aget water-row-cur j)
                                 (+ (aget terrain-row-cur j)
                                    (aget water-row-cur j)
                                    (- (aget energy-row-cur j))
                                    (- left-pressure)))
                            flow-div)]
                (aset water-row-prev j (+ (aget water-row-prev j) flow))
                (aset water-row-cur j (+ (aget water-row-cur j) (- flow)))
                (aset energy-row-prev j (* (aget energy-row-prev j) (- 1.0 friction)))
                (aset energy-row-prev j (+ (aget energy-row-prev j) (- flow)))))
            (if (and (< i (dec dim))
                     (> (+ (aget terrain-row-cur j)
                           (aget water-row-cur j)
                           (aget energy-row-cur j))
                        right-pressure))
              (let [flow (/ (min (aget water-row-cur j)
                                 (+ (aget terrain-row-cur j)
                                    (aget water-row-cur j)
                                    (aget energy-row-cur j)
                                    (- right-pressure)))
                            flow-div)]
                (aset water-row-next j (+ (aget water-row-next j) flow))
                (aset water-row-cur j (+ (aget water-row-cur j) (- flow)))
                (aset energy-row-next j (* (aget energy-row-next j) (- 1.0 friction)))
                (aset energy-row-next j (+ (aget energy-row-next j) flow))))
            (if (and (> j 0)
                     (> (+ (aget terrain-row-cur j)
                           (aget water-row-cur j)
                           (- (aget energy-row-cur j)))
                        back-pressure))
              (let [flow (/ (min (aget water-row-cur j)
                                 (+ (aget terrain-row-cur j)
                                    (aget water-row-cur j)
                                    (- (aget energy-row-cur j))
                                    (- back-pressure)))
                            flow-div)]
                (aset water-row-cur (dec j) (+ (aget water-row-cur (dec j)) flow))
                (aset water-row-cur j (+ (aget water-row-cur j) (- flow)))
                (aset energy-row-cur (dec j) (* (aget energy-row-cur (dec j)) (- 1.0 friction)))
                (aset energy-row-cur (dec j) (+ (aget energy-row-cur (dec j)) (- flow)))))
            (if (and (< j (dec dim))
                     (> (+ (aget terrain-row-cur j)
                           (aget water-row-cur j)
                           (aget energy-row-cur j))
                        front-pressure))
              (let [flow (/ (min (aget water-row-cur j)
                                 (+ (aget terrain-row-cur j)
                                    (aget water-row-cur j)
                                    (aget energy-row-cur j)
                                    (- front-pressure)))
                            flow-div)]
                (aset water-row-cur (inc j) (+ (aget water-row-cur (inc j)) flow))
                (aset water-row-cur j (+ (aget water-row-cur j) (- flow)))
                (aset energy-row-cur (inc j) (* (aget energy-row-cur (inc j)) (- 1.0 friction)))
                (aset energy-row-cur (inc j) (+ (aget energy-row-cur (inc j)) flow)))))
\end{lstlisting}


\subsection{Реконструкция поверхности}

В данном подразделе указан листинг, относящийся к реконструкции поверхности сцены.

В листинге \ref{lst:mesh} представлена реализация реконструкции поверхности.

\begin{lstlisting}[
                basicstyle={\fontsize{9}{10}\ttfamily},
                mathescape=true,
                showstringspaces=false,
                tabsize=2,
                flexiblecolumns=true,
                language={Clojure},
                label={lst:mesh}
            ]
(defn generate-voxel-mesh
  "Generate mesh that surrounds our voxel model"
  [voxels]
  (let
    [lookup-table (fill-lookup-table voxels)
     exposed-faces (find-exposed-faces voxels lookup-table)
     plane-faces (.entrySet exposed-faces)
     faces (pmap combine-faces-single-plane plane-faces)
     triangles (flatten (pmap triangulate-faces-single-plane faces))]

    (Mesh. triangles)))
\end{lstlisting}

\subsection{Рендеринг вокселов}

В данном подразделе указаны листинги, относящиеся к растеризации сцены.

В листинге \ref{lst:render} описана реализация растеризации воксельной сцены.

\begin{lstlisting}[
                basicstyle={\fontsize{9}{10}\ttfamily},
                mathescape=true,
                showstringspaces=false,
                tabsize=2,
                flexiblecolumns=true,
                language={Clojure},
                label={lst:render}
            ]
(defn draw-triangles
[canvas triangles mvp]
(let
[viewport [(.getWidth canvas) (.getHeight canvas)]
 width ^long (viewport 0)
 height ^long (viewport 1)
 neg-inf Double/NEGATIVE_INFINITY
 len ^long (* width height)
 z-buffer ^doubles (hiphip/amake Double/TYPE [_ len] neg-inf)]

(doall (pmap (fn [^Triangle triangle]
               (let [v1 ^Vertex (:v1 triangle)
                     v2 ^Vertex (:v2 triangle)
                     v3 ^Vertex (:v3 triangle)
                     p1 (camera/project-to-screen (:position v1) mvp viewport)
                     p2 (camera/project-to-screen (:position v2) mvp viewport)
                     p3 (camera/project-to-screen (:position v3) mvp viewport)
                     p1-x ^double (m/mget p1 0)
                     p1-y ^double (m/mget p1 1)
                     p1-z ^double (m/mget p1 2)
                     p2-x ^double (m/mget p2 0)
                     p2-y ^double (m/mget p2 1)
                     p2-z ^double (m/mget p2 2)
                     p3-x ^double (m/mget p3 0)
                     p3-y ^double (m/mget p3 1)
                     p3-z ^double (m/mget p3 2)
                     c ^int (colorize v2)
                     min-x ^long (long (max 0.0 (Math/ceil (min p1-x p2-x p3-x))))
                     max-x ^long (long (min (dec width) (Math/floor (max p1-x p2-x p3-x))))
                     min-y ^long (long (max 0.0 (Math/ceil (min p1-y p2-y p3-y))))
                     max-y ^long (long (min (dec height) (Math/floor (max p1-y p2-y p3-y))))
                     area ^double (+ (* (- p1-y p3-y) (- p2-x p3-x)) (* (- p2-y p3-y) (- p3-x p1-x)))]

                 (doall (p/pmap (fn [^long y]
                                  (doall
                                    (map (fn [^long x]
                                           (let [b1 ^double (/ (+ (* (- y p3-y) (- p2-x p3-x))
                                                                  (* (- p2-y p3-y) (- p3-x x)))
                                                               area)
                                                 b2 ^double (/ (+ (* (- y p1-y) (- p3-x p1-x))
                                                                  (* (- p3-y p1-y) (- p1-x x)))
                                                               area)
                                                 b3 ^double (/ (+ (* (- y p2-y) (- p1-x p2-x))
                                                                  (* (- p1-y p2-y) (- p2-x x)))
                                                               area)
                                                 depth ^double (+ (* b1 p1-z) (* b2 p2-z) (* b3 p3-z))
                                                 z-index ^long (+ (* y width) x)]
                                             (if (< (aget z-buffer z-index) (+ depth 10e-5))
                                               (do
                                                 (.setRGB canvas x y c)
                                                 (aset z-buffer z-index depth)))))
                                         (range min-x (inc max-x)))))
                                (range min-y (inc max-y))))))
             triangles))))

(defn draw-mesh [canvas ^Mesh mesh mvp]
(let [triangles (:triangles mesh)]+ 0.1
(draw-triangles canvas triangles mvp)))

(defn draw-voxels
[^BufferedImage canvas voxels ^Camera camera]
(let [mesh (voxel/generate-voxel-mesh voxels)
    mvp (camera/model-view-projection-matrix (camera/perspective (:position camera))
                                             (camera/get-view-matrix camera)
                                             model-matrix)]
(draw-mesh canvas mesh mvp)))
\end{lstlisting}

В листинге \ref{lst:lambert} описана реализация получения цвета по Ламберту.

\begin{lstlisting}[
                basicstyle={\fontsize{9}{10}\ttfamily},
                mathescape=true,
                showstringspaces=false,
                tabsize=2,
                flexiblecolumns=true,
                language={Clojure},
                label={lst:lambert}
            ]
(defn colorize [v1]
  (let [v1-xyz (:position v1)
        c (:color v1)
        r (.getRed c)
        g (.getGreen c)
        b (.getBlue c)
        v1-normal (vec/normalize (voxel/voxel-normal (:normal v1)))
        light-direction-v1 (vec/normalize (vec/sub @lights/light v1-xyz))
        view-direction-v1 (vec/normalize (vec/sub v1-xyz))
        reflect-v1 (vec/reflect (vec/sub light-direction-v1) v1-normal)
        diffuse-v1 (vec/scale lights/diffuse-albedo (max (vec/dot v1-normal light-direction-v1) 0.0))
        specular-v1 (vec/scale lights/specular-albedo (Math/pow (max (vec/dot reflect-v1 view-direction-v1) 0.0) lights/specular-power))
        modification (vec/add lights/ambient diffuse-v1 specular-v1)]

    (+ (bit-shift-left (int (* 255 (modification 3))) 24)
       (bit-shift-left (int (* r (modification 0))) 16)
       (bit-shift-left (int (* g (modification 1))) 8)
       (int (* b (modification 2))))))
\end{lstlisting}

% вывод по разделу
\section{Вывод}

В результате было реализовано программное обеспечение, моделирующее поведение воды.

Для написания ПО использован язык программирования Clojure, проведены оптимизации
кода с учётом подсказок типа.

%%% Local Variables:
%%% mode: latex
%%% TeX-master: "rpz"
%%% End:
