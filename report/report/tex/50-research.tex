\chapter{Экспериментально-исследовательский раздел}
\label{cha:research}

В данном разделе проводится апробация и анализ разработанной программы.
% скорость работы от параметров рендеринга
% другие метрики
% попробовать распараллелить

% листочки с презентацией подготовить
% демо программы

Анализ производится с помощью модуля Clojure - time, а также при помощи
VisualVM - инструмент профайлинга приложений под JVM.

Использовалось следующее аппаратное обеспечение:
\begin{itemize}
    \item процессор AMD Ryzen 5 3550h;
    \item ОЗУ DDR4 16GB.
\end{itemize}

Приложение запускалось под управлением операционной системы Ubuntu 19.10.

\section{Исследование характеристик программы}

Рассмотрим рендеринг сцены, состоящей из 30170 вокселов земли и 3000 вокселов воды
в разрешении $1920 \times 1080$.

Среднее время рендеринга одного кадра по результатам 140 измерений составило
3867 мс. При этом по результатам профайлинга утилитой VisualVM получено, что 65.3 процентов данного времени производились математические операции, причём утилизировано 627 потоков.

Таким образом, для ускорения рендеринга требуется использовать более мощное аппаратное обеспечение,
а математические операции передавать на GPU.

\section{Примеры использования программы}

В данном разделе представлены примеры использования разработанной программы.

Управление наблюдателем осуществляется либо с помощью интерфейса, либо с помощью
клавиш W, A, S, D и стрелок.

На \ref{screen:main} представлен снимок основного экрана программы.

\section{Выводы}

Программа работоспособна и оптимизирована с использованием ранее выбранных технологий.
Анализ показал, что математические операции следует исполнять на более мощном аппаратном обеспечении,
чем то, на котором производилось исследование.

%%% Local Variables:
%%% mode: latex
%%% TeX-master: "rpz"
%%% End:
