\chapter{Аналитический раздел}
\label{cha:analysis}
%
% % В начале раздела  можно напомнить его цель
% IDEF0


В данном разделе производится анализ методов разложения в растр объёмных сцен
с помощью вокселей.

Далее рассматриваются идеи применения данных методов к симуляции жидкостей и существующие решения в этой области.

% Обратите внимание, что включается не ../dia/..., а inc/dia/...
% В Makefile есть соответствующее правило для inc/dia/*.pdf, которое
% берет исходные файлы из ../dia в этом случае.

\section{Физическая модель}

Жидкости моделируются как векторное поле скорости жидкости и скалярное поле
плотности. Движение задаётся уравнениями Навье-Стокса \cite{inbook:bigenc}.

Далее будем рассматривать только движение воды (несжимаемой жидкости) в условиях постоянной температуры.

Тогда уравнения Навье-Стокса в векторной форме принимают следующий вид:
\begin{equation}
    \label{eq:navier-stokes}
    \frac{\partial \vec{v}}{\partial t} + \vec{v} \cdot \nabla\vec{v}= \vec{F} - \frac{1}{\rho} \nabla p + \eta\Delta\vec{v}.
\end{equation}

В уравнении \ref{eq:navier-stokes} $\vec{v}$ - скорость частицы воды,
                                   $t$ - время,
                                   ${\vec{F}}$ - внешняя удельная сила,
                                   $p$ - давление,
                                   $\eta = \frac{\mu}{\rho}$ - кинематический коэффициент вязкости,
                                   $\nabla$ - оператор Гамильтона,
                                   $\Delta$ - оператор Лапласа.

\section{Существующие подходы к симуляции жидкостей}



\section{Анализ воксельных методов разложения в растр}


% вывод по разделу

\section{Вывод}


%%% Local Variables:
%%% mode: latex
%%% TeX-master: "rpz"
%%% End:
