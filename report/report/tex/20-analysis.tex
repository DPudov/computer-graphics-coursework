\chapter{Аналитический раздел}
\label{cha:analysis}
%
% % В начале раздела  можно напомнить его цель
% IDEF0


В данном разделе производится анализ методов вычислений характеристик жидкостей и
их преобразований в графический формат.

Далее рассматриваются идеи применения данных методов к симуляции жидкостей и существующие решения в этой области.

% Обратите внимание, что включается не ../dia/..., а inc/dia/...
% В Makefile есть соответствующее правило для inc/dia/*.pdf, которое
% берет исходные файлы из ../dia в этом случае.

\section{Физическая модель}

Жидкости моделируются как векторное поле скорости жидкости и скалярное поле
плотности. Движение задаётся уравнениями Навье-Стокса \cite{inbook:bigenc}.

Далее рассматривается только движение воды (несжимаемой жидкости) в условиях постоянной температуры.

Тогда уравнения Навье-Стокса в векторной форме принимают следующий вид:
\begin{equation}
    \label{eq:navier-stokes}
    \frac{\partial \vec{v}}{\partial t} + \vec{v} \cdot \nabla\vec{v}= \vec{F} - \frac{1}{\rho} \nabla p + \eta\Delta\vec{v}.
\end{equation}

В уравнении \ref{eq:navier-stokes} $\vec{v}$ - скорость частицы воды,
                                   $t$ - время,
                                   ${\vec{F}}$ - внешняя удельная сила,
                                   $p$ - давление,
                                   $\eta = \frac{\mu}{\rho}$ - кинематический коэффициент вязкости,
                                   $\nabla$ - оператор Гамильтона,
                                   $\Delta$ - оператор Лапласа.

Данная физическая модель лежит в основе многих подходов симуляции жидкостей\cite{book:ash}. Их
обзор приведён далее.

В статистической физике модель поведения частиц жидкости описывается кинетическим
уравнением Больцмана. Данная модель применима для систем, где
есть ограничения на малую скорость частиц\cite{site:bolzman}.

\section{Существующие подходы к симуляции жидкостей}

В вычислениях поведения жидкости необходимо представить физическую модель в
 дискретном виде. Данную проблему решает вычислительная гидродинамика - совокупность
 теоретических, экспериментальных и численных методов, предназначенных для моделирования
 потоковых процессов.

Наиболее распространёнными методами описания характеристик жидкости в
вычислительной гидродинамике являются:
\begin{itemize}
    \item сеточные методы Эйлера;
    \item метод гидродинамики сглаженных частиц;
    \item методы, основанные на турбулентности;
    \item метод решёточных уравнений Больцмана.
\end{itemize}

Сеточные методы Эйлера являются наиболее простым решением симуляции жидкостей
Они заключаются в поиске решения задачи Коши для функций,
заданных таблично. Для каждого узла функции уровня жидкости
 требуется вычисление значений разложений Тейлора в их окрестностям. Решение задачи Коши
 в данном случае аппроксимирует решение уравнения Навье-Стокса\cite{book:compmath}.

Метод гидродинамики сглаженных частиц и методы, основанные на турбулентности,
заключаются в выборе размера частицы ("длины сглаживания"), на котором их свойства
"сглаживаются" посредством функции ядра или интерполяции, и решения уравнений
Навье-Стокса с учётом вязкости и плотности. Это позволяет эффективно моделировать
поведение жидкостей, газов и даже использовать в астрофизике\cite{site:astro}.

 Метод решёточных уравнений Больцмана основан на кинетическом уравнении Больцмана,
 упомянутом ранее. Этот метод поддерживает многофазные жидкости, наличие теплопроводности
 и граничные условия на макроскопическом уровне\cite{site:habr-physics}.

\section{Анализ методов визуализации жидкостей}

Основными требованиями к методам симуляции жидкостей со стороны компьютерной графики
являются визуальная правдоподобность и скорость анимации.

Исходя из этих условий, в компьютерной графике используются специально модифицированные
и оптимизированные методы, основанные на указанных ранее.

В наше время существует всего 2 принципиально различных техники объёмного рендеринга:
трассировка лучей и реконструкция поверхности.

\subsection{Трассировка лучей}

Данная техника состоит в том, чтобы создать виртуальный экран и далее проследить
путь луча до тех пор, пока он проходит сквозь анимируемые объекты. В итоге луч
позволяет определить цвет каждого отдельного пикселя. Настоящий алгоритм трассировки лучей
имеет значительную вычислительную сложность, поэтому его часто заменяют на
правдоподобные аппроксимации.

Одним из способов аппроксимации является трассировки лучей путём
замены полигонов на воксели. Это позволяет существенно повышать производительность
анимации за счёт возможностей параллелизма и снижения вычислительной сложности.
С другой стороны для хранения вокселей требуется большее количество
памяти, чем для хранения полигонов.

Другим популярным методом является метод срезов объёмных текстур так же, как и воксельный, является математическим упрощением алгоритма трассировки лучей.
Данные сцены представляются в виде 3D текстуры. Рендеринг производится с самого дальнего
 по отношению к наблюдателю среза текстуры. В итоге видимыми являются ближайшие из частей
 среза текстуры.
 В качестве составляющих срезов могут быть использованы как воксели, так и полигоны.
 Этот способ оптимизирован аппаратно для некоторых графических процессоров
 и в этом случае имеет преимущество перед воксельным методом по времени работы.

\subsection{Реконструкция поверхности}

Реконструкцией поверхности, или трёхмерной реконструкцией, называют процесс
получения облика реальных объектов.

В качестве входных данных получают множество точек, характеризующее объект.
Далее выбирают точки на поверхности объекта так, что получают полигоны, которые
имеют необходимую конфигурацию для аппроксимации формы поверхности в её видимой части.
Набор результирующих полигонов - реконструированная модель.

Основным алгоритмом, реализующим этот метод, является Marching cubes.
Он широко используется в компьютерной и магнитно-резонансной томографии.
Суть алгоритма в том, что для кубов, находящихся на поверхности объекта, известно,
какие точки лежат внутри объекта и какие снаружи. Это позволяет выбрать
приближающий полигон так, чтобы его вершины находились на отрезках, соединяющих вершины куба,
в примерных точках реального пересечения с объектом.
Таким образом, однажды применив алгоритм, можно использовать объект в анимации, если
он не изменяет форму поверхности.
Для предметов, теряющих форму, например, жидкостей требуется повторное вычисление
реконструированной поверхности. Подобные вычисления имеют значительную сложность,
поэтому на практике не используются.

% вывод по разделу

\section{Вывод}


%%% Local Variables:
%%% mode: latex
%%% TeX-master: "rpz"
%%% End:
